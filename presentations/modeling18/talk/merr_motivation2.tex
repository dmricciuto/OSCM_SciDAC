% !TEX root = slides.tex

%%%%%%%%%%%%%%%%%%%%%%%%%%%%%%%%%%%%%%%%%%%%%%%%%%%%%%%%%%%%%%%%%%%%%%%%%%%%%%
\begin{frame}[t]
\label{motivation}

\frametitle{Main target: model \emph{structural} error \\ \small{ deviation from `truth' or from a higher-fidelity model}}

\bbi
\item Inverse modeling context
% \hfill \begin{tikzpicture} \node [rounded corners,fill=blue!20] {
% Data meets Models};
% \end{tikzpicture}
\bgi
\item Given experimental or higher-fidelity model data,\\
estimate the model error
\egi
\medskip
\hrule
\medskip
\item Represent and estimate the error associated with
\bgi
\item Simplifying assumptions, parameterizations
\item Mathematical formulation, theoretical framework
%\item Numerical discretization
\egi
\medskip
\item ...will be useful for
\bgi
\item Model validation
\item Model comparison
\item Scientific discovery and model improvement
\item Reliable computational predictions
\egi

\ebi
\end{frame}
% %%%%%%%%%%%%%%%%%%%%%%%%%%%%%%%%%%%%%%%%%%%%%%%%%%%%%%%%%%%%%%%%%%%%%%%%%%%%%%
% \begin{frame}[t]
% \label{motivation1}

% \frametitle{\hspace*{2.5cm} Data informs model parameters: \\ \hspace*{2.5cm} \small{but what if the model is only an approximation?}}

% \centerline{
% \includegraphics<1>[width=0.8\textwidth,clip]{merr/{model2}.eps}
% \includegraphics<2>[width=0.8\textwidth,clip]{merr/{model1}.eps}
% \includegraphics<3>[width=0.8\textwidth,clip]{merr/{model12}.eps}
% \includegraphics<4>[width=0.8\textwidth,clip]{merr/{model12_}.eps}
% }

% \end{frame}


%%%%%%%%%%%%%%%%%%%%%%%%%%%%%%%%%%%%%%%%%%%%%%%%%%%%%%%%%%%%%%%%%%%%%%%%%%%%%%
\begin{frame}[t]
\label{motivation3}

\frametitle{\hspace*{2.5cm} Ignoring model error leads to \\ \hspace*{2.cm}overconfident and biased predictions}
\vspace{-0.3cm}
\only<1-6>{
\begin{tabular}{ p{0.55\textwidth}  p{0.45\textwidth}  }
\only<1>{\includegraphics[width=0.5\textwidth,clip]{merr/test_sketch/{data_only_5}.eps} & }
\only<2>{\includegraphics[width=0.5\textwidth,clip]{merr/test_sketch/{data_fit_5}.eps} & \includegraphics[width=0.33\textwidth,clip]{merr/test_sketch/{post_5}.eps} }
\only<3>{\includegraphics[width=0.5\textwidth,clip]{merr/test_sketch/{data_fit_model_5}.eps} & \includegraphics[width=0.33\textwidth,clip]{merr/test_sketch/{post_5}.eps} }
\only<4>{\includegraphics[width=0.5\textwidth,clip]{merr/test_sketch/{data_fit_model_20}.eps} & \includegraphics[width=0.33\textwidth,clip]{merr/test_sketch/{post_20}.eps} }
\only<5>{\includegraphics[width=0.5\textwidth,clip]{merr/test_sketch/{data_fit_model_50}.eps} & \includegraphics[width=0.33\textwidth,clip]{merr/test_sketch/{post_50}.eps} }
\only<6>{\includegraphics[width=0.5\textwidth,clip]{merr/test_sketch/{data_fit_model_50}.eps} & \hspace*{-0.8cm}\includegraphics[width=0.5\textwidth,clip]{merr/test_sketch/{data_fit_model_50_abc}.eps} }
\end{tabular}
\small
\only<1-5>{\hspace*{2.cm} Model-data fit} \only<6>{\hspace*{1.2cm} No model error treatment} \only<2-5>{\hspace*{30mm} Posterior on parameters} \only<6>{\hspace*{2.cm} Model error accounted for}
}
\vspace*{2mm}
\footnotesize
\bi \setlength{\leftmargini}{-2.5pt}
\only<1->{\item Given noisy data, calibrate an exponential model:  $\quad g(x)\approx f(x;\lambda)$} %, $\textrm{ i.e. } y={f(x;\lambda)}+\epsilon$
\only<2->{\item Employ Bayesian inference to obtain posterior PDFs on $\lambda$}
\only<3-6>{\item True model -- dashed-red -- is \emph{structurally} different from fit model $f(x,\lambda)$}
\only<4-5>{\item Higher data amount reduces posterior and predictive uncertainty
\bri
\item Increasingly sure about predictions based on the \emph{wrong} model
%\item If the model has structural uncertainty, more data leads to biased and overconfident results
\eri
}
\only<6>{\item Accounting for model error allows extra uncertainty component to propagate through predictions}
% \item We want to quantify model-vs-truth discrepancy in a rigorous and systematic way
% \item Cannot ignore model error
\ei

\end{frame}
%%%%%%%%%%%%%%%%%%%%%%%%%%%%%%%%%%%%%%%%%%%%%%%%%%%%%%%%%%%%%%%%%%%%%%%%%%%%%%

